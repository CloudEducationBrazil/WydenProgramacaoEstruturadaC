-- Configuração Windows
http://cienciacomputacao.com.br/tutorial/instalando-a-biblioteca-mysql-no-codeblocks/  

//para compilar faça "gcc -o example `pkg-config --cflags --libs gtk+-2.0` example.c"

-- Dependências MYSQL Linguagem C - Biblioteca
https://dev.mysql.com/downloads/connector/c/

https://dev.mysql.com/doc/dev/mysql-server/latest/mysql_8h.html

https://www.vivaolinux.com.br/artigo/Usando-MySQL-na-linguagem-C

https://www.hardware.com.br/comunidade/conectando-mysql/975541/

https://www.youtube.com/watch?v=_yKEtJhwVYc

https://forum.imasters.com.br/topic/476781-conectar-a-base-de-dados/

https://www.vivaolinux.com.br/artigo/Usando-MySQL-na-linguagem-C

Só para colaborar, no debian precisei instalar o libmysqlclient-dev para ter o mysql.h
sudo apt-get install libmysqlclient-dev

http://www.ufjf.br/jairo_souza/files/2009/12/3-Arquivos-Manipula%C3%A7%C3%A3o-de-arquivos-em-C.pdf

fopen() Abre um arquivo
Fclose () Fecha um arquivo
putc() e fputc() Escreve um caractere em um arquivo
getc() e fgetc() Lê um caractere de um arquivo
fseek() Posiciona em um registro de um arquivo
fprintf() Efetua impressão formatada em um arquivo
fscanf() Efetua leitura formatada em um arquivo
feof() Verifica o final de um arquivo
fwrite() Escreve tipos maiores que 1 byte em um arquivo
fread() Lê tipos maiores que 1 byte de um arquivo

Modos de abertura

Modo Significado
"r" Abre um arquivo texto para leitura. O arquivo deve existir antes de ser aberto.
"w" Abrir um arquivo texto para gravação. Se o arquivo não existir, ele será criado. Se já existir, o conteúdo anterior será destruído.
"a" Abrir um arquivo texto para gravação. Os dados serão adicionados no fim do arquivo ("append"), se ele já existir, ou um novo arquivo será
criado, no caso de arquivo não existente anteriormente.
"rb" Abre um arquivo binário para leitura. Igual ao modo "r" anterior, só que o arquivo é binário.
"wb" Cria um arquivo binário para escrita, como no modo "w" anterior, só que o arquivo é binário.
"ab" Acrescenta dados binários no fim do arquivo, como no modo "a" anterior, só que o arquivo é binário.
"r+" Abre um arquivo texto para leitura e gravação. O arquivo deve existir e pode ser modificado.
"w+" Cria um arquivo texto para leitura e gravação. Se o arquivo existir, o conteúdo anterior será destruído. Se não existir, será criado.
"a+" Abre um arquivo texto para gravação e leitura. Os dados serão adicionados no fim do arquivo se ele já existir, ou um novo arquivo será
criado, no caso de arquivo não existente anteriormente.
"r+b" Abre um arquivo binário para leitura e escrita. O mesmo que "r+" acima, só que o arquivo é binário.
"w+b" Cria um arquivo binário para leitura e escrita. O mesmo que "w+" acima, só que o arquivo é binário.
"a+b" Acrescenta dados ou cria uma arquivo binário para leitura e escrita. O mesmo que "a+" acima, só que o arquivo é binário


Gravando e lendo Dados em Arquivos

 Existem várias funções em C para a operação de
gravação e leitura de dados em arquivos. Abaixo
seguem algumas:
 putc() ou fputc(): Grava um único caracter no arquivo
 fprintf() : Grava dados formatados no arquivo, de acordo
com o tipo de dados (float, int, ...). Similar ao printf,
porém ao invés de imprimir na tela, grava em arquivo
 fwrite() : Grava um conjunto de dados heterogêneos
(struct) no arquivo
 fscanf(): retorna a quantidade variáveis lidas com
sucesso



